\documentclass[xcolor=dvipsnames]{beamer}
\usepackage{graphicx} 
\usepackage{tabularx}
\usepackage{array}
\usepackage{booktabs} 
\mode<presentation>{
    
\definecolor{MSUgreen}{RGB}{0,100,11} 
\usetheme{Boadilla}
\usecolortheme{spruce}
\setbeamercolor{block title}{bg=MSUgreen!10, fg=MSUgreen!60!black}
\setbeamercolor{navigation symbols dimmed}{fg=MSUgreen!40!white}
\setbeamercolor{navigation symbols}{fg=MSUgreen!40!white}
\setbeamertemplate{itemize item}{\color{MSUgreen}\textbullet}
\setbeamercolor{section in toc}{fg=MSUgreen}
\setbeamercolor{section number projected}{bg=MSUgreen,fg=white}
\setbeamercolor{subsection in toc}{fg=MSUgreen}
\setbeamercolor{subsection number projected}{bg=MSUgreen,fg=white}
}

\graphicspath{ {.././image/} }

\title[CostCompiler]{Un Prototipo per lo scheduling di funzioni basato su analisi di costo in piattaforme serverless}
\subtitle{Sviluppo di un interprete per l'analisi di costo di funzioni serverless}
\author{Simone Boldrini}
\date{14 Marzo 2024}
\institute[]{Alma Mater Studiorum - Università di Bologna \\ Facoltà di Scienze}

\begin{document}


\begin{frame}
    \titlepage
\end{frame}

\begin{frame}
    \frametitle{Introduzione}
    \alert{Obiettivo}: Sviluppare un prototipo di compilatore per piattaforme serverless che sfrutti tecniche di analisi di costo per ottimizzare l'esecuzione di funzioni.
    \begin{itemize}
        \item<1-> Definizione grammatica specifica 
        \item<2-> Analisi del programma
        \item<3-> Generazione equazioni di costo 
        \item<4-> Generazione del codice WASM
    \end{itemize}
\end{frame}
\begin{frame}
    \frametitle{Definizione della grammatica}
    Abbiamo definito una grammatica specifica $HLCostLan$ per la defizione di un linguaggio di alto livello per la definizione di funzioni serverless.
\end{frame}
\begin{frame}
    \frametitle{Analisi del programma}
    Una volta definito il linguaggio, abbiamo sviluppato un interprete per l'analisi del programma.
    Quest'analisi prevede:
    \begin{itemize}
        \item<1-> Analisi lessicale e sintattica(Riconosciuta da ANTLR)
        \item<2-> Analisi semantica 
    \end{itemize}
\end{frame}
\begin{frame}
    \frametitle{Generazione equazioni di costo}
    Una volta analizzato il programma, abbiamo sviluppato un interprete per la generazione delle equazioni di costo.
\end{frame}
\begin{frame}
    \frametitle{Analisi di costo}
    
    \newtheorem{Analisi di costo}{Analisi di costo}

    \begin{Analisi di costo}
        Come \alert{analisi statica dei costi} miriamo ad ottenere risultati analitici per un dato programma $P$, i quali consentono di vinciolare il costo dell'esecuzione di $P$ su qualsiasi input $x$, senza dover effettivamente eseguire $P(x)$.
    \end{Analisi di costo}
   
    PUBS ha l'obiettivo di ottenere automaticamente un upper bound in forma chiusa per i sistemi di equazioni di costo, calcolando i limiti superiri per la relazione di costo indicata come ``entry'', oltre che per tutte le altre relazioni da cui tale ``entry'' dipende.
\end{frame}
\begin{frame}
    \frametitle{Analisi di costo}
    Un'analisi di costo è fortemente dipendente dal modello di costo preso in considerazione:
    \begin{itemize}
        \item \alert{Costo di esecuzione}: il costo di esecuzione di una funzione
        \item \alert{Costo di allocazione}: il costo di allocazione di una variabile nell'heap
    \end{itemize}
    I vantaggi delle equazioni di costo:
    \begin{itemize}
        \item Sono \alert{indipendenti} dal linguaggio di programmazione
        \item Possono rappresentare diverse classi di \alert{complessità}
        \item Possono catturare una varietà di nozioni non banali di risorse.
    \end{itemize}
\end{frame}
\begin{frame}
    \frametitle{Generazione del codice WASM}
    Una volta ottenute le equazioni di costo, abbiamo sviluppato un interprete per la generazione del codice WASM.
\end{frame}

\end{document}